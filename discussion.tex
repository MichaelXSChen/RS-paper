\section{Discussions}\label{sec:discuss}

This section discusses \xxx's limitations (\S\ref{sec:limits}) and its broad 
applications (\S\ref{sec:apps}).

\subsection{Limitations}\label{sec:limits}

% P1: Soundness and completeness
% State the main goal of our tool: scalable to large code and accurate. Neither
% sound or complete.
\xxx's main design goal is to achieve reasonable accuracy and scalability, 
and it trades off soundness (\ie, do not miss any bugs), although evaluation 
results show that our \xxx tool can already effectively detect \nbackendDetected out of \nreproduced 
concurrency attacks. A typical way to ensure soundness is to plug in a sound 
alias analysis tool (\eg,~\cite{bddbddb, dsa:pldi07}) to identify all LLVM 
\v{load} and \v{store} instructions that may access the same memory. However, 
typical alias analyses are known to be inaccurate (\eg, often report too 
many false positives). \xxx can leverage advanced precise alias analysis 
techniques~\cite{peregrine:sosp11, wu:pldi12, chimera:pldi12} to improve 
soundness, and we leave it for future work.

% P2: Vulnerability site; limited, external functions.
\xxx currently handles five types of regular vulnerability operations 
(\S\ref{sec:patterns}), and it requires these operations to exist in the LLVM 
bitcode. These five types of operations are sufficient to cover all 
\nreproduced concurrency attacks we have reproduced, and more types can be 
added. If developers are concerned about some third-party code that may contain 
vulnerabilities, they should compile this code into the bitcode for \xxx.

% P3: Static, can not figure out the exact value without actual executions. For 
% example, branches taken to avoid vulnerabilities, pointer dereference values 
% depending on inputs.
\xxx's consequence analysis tool integrates call stack of a concurrency bug to 
direct static analysis toward vulnerable program paths, but static analysis can 
still report false positives because it lacks actual execution outcomes 
(\eg, branches taken to avoid vulnerabilities, and pointer dereference values 
depending on inputs). Developers can inspect the propagation chains output by
\xxx's analyzer and refine their program inputs to validate the outcomes.




% False negative:
% 1. No alias analysis. Focus on describing the limitations of alias: too many 
% false positives, false negatives, and buggy.


% False positive:
% 1. library functions that couldn't obtain source code.
% 2. data flow analysis is path insensitive
% 3. data flow analysis is flow insensitive
% 4. Static, can not figure out the exact value without actual executions. For 
% example, branches taken to avoid vulnerabilities, pointer dereference values 
% depending on inputs.

\subsection{\xxx has Broad Applications}\label{sec:apps}

% Spur new tools for concurrency attacks. Work with anormaly detections. Work 
% with process race detections.
We envision three applications for \xxx's techniques. First, \xxx can 
spur new concurrency attack defense techniques by compensating various runtime 
tools. For instance, we can leverage anomaly detection~\cite{schonberg:pldi89, 
taskrecycling:ppopp90,diduce:icse02} and intrusion 
detection~\cite{hofmeyr:syscall:ids, wagner:syscall:ids, wagner:intrusion} tools 
to audit only the vulnerable program paths identified by \xxx, then these 
runtime detection tools can greatly reduce the amount of program paths that need 
to audit and improve performance. \xxx can also integrate with other bug 
detection tools (\eg, process races~\cite{racepro:sosp11} and atomicity 
bugs~\cite{ctrigger:asplos09}) to detect concurrency attacks caused by such 
bugs.

Second, \xxx's consequence analysis tool has the potential to detect various 
consequences of software bugs. Software bugs have 
caused many extreme disasters~\cite{10-historical-bug-consequences, 
10-epic-software-bugs} in the last few decades, including losing big money and 
taking lives. By adding new vulnerability and failure sites of such 
consequences, \xxx can be applied to flagging bugs that can cause severe 
consequences among the enormous bug reports.

Third, \xxx's techniques can be applied to detect general concurrency bugs. 
Both static and dynamic concurrency bug detectors are prone to false positives, 
and inspecting these reports have caused too much manpower and money. 
\xxx's false positive filters and analysis tool may mitigate these issues.
