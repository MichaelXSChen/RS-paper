\section{Discussions}\label{sec:discuss}

This section discusses \xxx's limitations (\S\ref{sec:limits}) and its 
applications in other research areas (\S\ref{sec:apps}).

\subsection{Limitations}\label{sec:limits}

% Clamav style.
\xxx's output checking protocol may have false positives or false negatives, 
because it just makes an effort to practically indicate that replicas are 
running the same execution states. A server program running in \xxx may have 
false positive when it uses multiple threads to serve the same client 
connection and uses these threads to concurrently produce outputs 
(\eg, \clamav). Running a DMT scheduler in \xxx can address this problem. In 
our evaluation, all programs except \clamav uses only one thread to process one 
client connection and they don't have such false positives.

A server program may also have false negative when it triggers a software bug 
but the bug does not propogate to network outputs. On client programs' point of 
view, such bugs do not matter; \xxx already checkpoints file system state to 
mitigate this issue.

When execution divergence is detected in a replica, \xxx's rollback mechanism 
is not designed to guarantee that the re-executions of this replica will 
definitely avoids this divergence. We made this design choice because both 
our evaluation and a previous work Eve~\cite{eve:osdi12} found that divergence 
happens extremely rarely in evaluation. Specifically, although Eve provided a 
sequential re-execution approach to with divergence avoidance guarantee, which 
\xxx can leverage, but even Eve's evaluation didn't experience any divergence 
and thus this approach was not in use at all.

% Have not hooked time and rand(). Can use PMP approach. Not a problem for 
% evaluated apps. 
\xxx currently does not hook random functions such as \v{gettimeofday()} and 
\v{rand()} because our output checkers have not detected network output 
divergence comming from these functions. Existing 
approaches~\cite{eve:osdi12,paxos:practical} in \paxos protocols can be 
leveraged to intercept these functions and make them produce same results among 
replicas.

% No guarantees on recovery. Best effort. But, can we actually server requests 
% one by one?

%% No lxc.


\subsection{Applications in Broad Research Areas}\label{sec:apps}

% Other replication topics.

% Parallel program analysis. Such a short coordination time can make replicas 
% run almost as fast as each other, support many time-critical analyses 
% such as race detection and security defenses.

% One idea: can leverage the checkpoint and rollback protocok, and the 
% consensus part, to build a system that can automatically bypass concurrency 
% bugs without fixing them. The way to bypass: proactively reordering socket 
% calls. Self-healing system.

% A core building block in future operating systems. Maintain a consistent view 
% of computing resources and data. Used in scheduling framework (Mesos) because 
% its latency is almost compatible with context switches of processes (sub 
% milli seconds). 
