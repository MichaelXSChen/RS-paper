\section{Discussions} \label{sec:discussion}

This section first discusses \xxx's limitations and then introduces its 
applications.

\subsection{Limitation} \label{sec:limit}

% \xxx's key benefit is that a server program's blocking socket operations and 
% \pthread synchronization operations return at the same logical clocks across 
% different replicas, if a server program has no data race, no ad-hoc 
% synchronization, and does not have nondeterministic functions (\eg, 
% \v{rand()}).

\xxx leverages \parrot to make synchronizations deterministic.  \parrot is
explicitly designed not to handle data races. However, in the context of \xxx,
data races are less harmful because, if they cause backups to crash, \xxx
can still operate and recover as long as a quorum of the replicas is
still alive. Moreover, leveraging \xxx's replication architecture, one can 
deploy a race detector on a backup replica~\cite{repframe:apsys15}, achieving 
both good \xxx performance and full determinism.

There are other sources of nondeterminism besides thread scheduling and
request timing.  These other sources of nondeterminism may cause backups
to diverge, too.  For example, backups may do different things based on
their IP addresses, data read from \v{/dev/random}, addresses returned by
\v{malloc}, physical time observed via \v{gettimeofday}, or delivery time
of signals.  Prior work has shown how to eliminate these sources of
nondeterminism using record-replay~\cite{scribe:sigmetrics2010, 
respec:asplos10} or OS-level techniques~\cite{dos:osdi10}, which \xxx can 
leverage.  Another solution is to treat all these sources as inputs and 
leverage distributed consensus to let all replicas observe the same input.  We 
leave these ideas for future work. We inspected server 
programs' network outputs among replicas, and we found that these outputs 
were consistent in \xxx except physical times (\S\ref{sec:correctness}).

For a server program that spawns multiple processes which communicate via
IPC, \xxx currently does not make these IPC operations deterministic.  We
expect that it should be easy to support deterministic IPC in \xxx because it
already makes socket API deterministic.  In addition,
dOS~\cite{dos:osdi10} and DDOS~\cite{ddos:asplos13} have many effective
techniques for tackling this problem, which \xxx can leverage.

\subsection{Applications} \label{sec:app}

We envision three applications for \xxx. First, \xxx can 
be leveraged by other replication concepts (\eg, byzantine fault 
tolerance~\cite{pbft:osdi99, zyzzyva:sosp07}) 
and record-replay~\cite{scribe:sigmetrics10, racepro:sosp11, respec:asplos10} 
because they also suffer from nondeterminism. Second, promising 
results in \repframe~\cite{repframe:apsys15} have shown that \xxx's transparent 
replication architecture can enable multiple types of program analysis tools 
within one execution, making a server program enjoy benefits of multiple 
analyses. Third, \xxx's determinism as well as its \timealgo technique alone can 
be applied to mitigate timing channels~\cite{Askarov:ccs10, Zhang:ccs11, 
Aviram:ccsw10}.