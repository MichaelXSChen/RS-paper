\section{Input Coordination Protocol} \label{sec:input}

This section first introduces the workflow of \xxx input coordination protocol 
in normal case, and describes how to elect a new leader for exceptional 
cases.

\subsection{Normal Case Operations} \label{sec:normal}


\subsubsection{Primitives} \label{sec:primitive}

% Question: What if on a machine, it viewed it as the leader, it executes the 
% actual socket call, and then it invokes the consensus, but then it is no 
% longer the leader any more. How to undo the actual socket calls? Cases:
  %% On accept(). The old leader has already accepted and asks for consensus.
  %% On recv(). The old leader has already received.
  %% On send().
  %% On close().

% First, basic roles. log format details.

% Second, leader thread behavior on \recv(). Return from orig recv, grab 
% spin % lock and get view stamp, store the log, and post send to all replicas. 
% Non blocking. Wait % for over half agree, and then execute the operation.

% Post send carry the last committed (consensus reached) operation.

% One key issue, check integrity. Strawmen approach, check viewstamp first.

% Third, replica thread behavior on \recv(). Block on latest un agreed log, 
% wait for the % consensus log, check integrity, check view id, and then store to 
% BDB. and % then execute the committed but not executed requests from BDB.

\subsection{Handling Concurrent Connections} \label{sec:concurrent}

% TBD. Strawmen approach, use fds on leader machines.

% Use viewstamp of the accept() operation, consistent across replicas.

% What does replica thread do. High performance loop on a dedicated core.

\subsection{Leader Election} \label{sec:election}

% % Use one log operation or a few operations to do so. Clean, 
% match the PMS intention. Explain why this design choice is good.

% TBD. Four steps. 