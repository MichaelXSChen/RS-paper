
% S1: SMR, reliable, good for online services, machines, network. 
% S2: but going through network round trips for requests hurt latency of 
% services.
% S3: Traditional network layers or OS are not inherent to consensus protocols 
% because these protocols have already considered exceptional cases such as 
% packet losses and kernel failures.
% S4: We can leverage RDMA (one side read/write).
% S5: \xxx, a speed input coordination protocol that levearages fastest RDMA 
% operations.
% \xxx addresses an important challenge that practically enforce execution 
% states on systems nondeterminism such as thread nondeternimism and systems 
% resources contentions.

% State machine replication (SMR) runs replicas of the same program and 
% uses a distributed consensus protocol (\eg, \paxos) to enforce the same inputs 
% among replicas, tolerating various faults. Although recent SMR systems have 
% greatly improved the reliability of server programs, SMR remains difficult to 
% deploy due to its high consensus latency. This paper presents \xxx, 
% a fast SMR system by leveraging Remote Direct Memory Access (RDMA). \xxx 
% intercepts inputs from a server program's socket calls and runs a new 
% RDMA-accelerated \paxos protocol to coordinate these inputs among replicas. This 
% protocol addresses a \paxos scalability challenge by tightly integrating RDMA 
% features within the fault-tolerant nature of \paxos, making replicas reach 
% consensus rapidly in parallel. Leveraging this protocol, \xxx presents a fast 
% network output checking protocol to check whether replicas run in sync.

Consensus protocol (typically, \paxos) enforce a strongly consistent order of 
inputs for the same program replicated on a group of machines (or replicas), 
tolerating various failures. Therefore, \paxos is widely 
served in numerous systems, including ordering and fault-tolerance services. 
Unfortunately, despite much effort, the group size of traditional \paxos 
protocols can hardly go up to a dozen, because traditionally consensus messages 
go throught TCP/IP and thus consensus latency will increase almost linearly to 
the replica group size. This paper presents \xxx, a fast, scalable \paxos 
runtime system by fully exploiting Remote Direct Memory Access (RDMA) features. 
Our key idea to achieve scalability is making \paxos replicas receive consensus 
messages \emph{purely on local memory}, just like making threads communicate 
with other threads efficiently via bare memory.

\xxx is the first to achieve low \paxos consensus latency with high 
scalability (100+ replicas). Evaluation shows that the \xxx's consensus latency 
was faster than four popular \paxos implimentations by \comptradlow to 
\comptradhigh. When increasing the replica group size from three to 105, \xxx's 
consensus latency increases merely from \xxxlatencythree to 
\xxxlatencyonezerofive. \xxx is faster than a recent RDMA-based \paxos protocol 
by up to \fasterDARE. It was able to support 9 widely used, unmodified server 
programs (\eg, \redis and \mysql) with low overhead. All \xxx source code, 
benchmarks, and raw evaluation results are available at \github.

% \xxx's key to achieve scalability is making \paxos replicas receive consensus 
% messages \emph{purely on local memory}, creating an ilusion that these replicas 
% communicate via shared-memory on multi-core.

% Our key idea to achieve 
% scalability is making \paxos replicas receive consensus messages \emph{purely 
% on local memory}, totally getting rid of traditional TCP/IP communication 
% primitives as well as RDMA inbound communication primitives. 

% on \nprog widely used, 
% diverse server programs (\eg, \memcached and 
% \mysql) shows that \xxx is: (1) general, it ran these servers 
% without modifications except one program; (2) fast, it incurred merely a 
% \latencyoverhead mean overhead in response time and \tputoverhead in 
% throughput, and its consensus latency was \fasterthanzookeeper faster than 
% a prior SMR system built on \zookeeper; (3) scalable, it achieved similar 
% consensus latency on a replica group size of three to seven; and (4) 
% robust, its network output protocol efficiently detected and recovered replicas 
% from divergence. 
% \xxx addresses 
% a pervasive 
% challenge, avoiding a server' execution state divergence in active replicas, by 
% presenting a fast, application-agnostic output checking mechanism on top of our 
% consensus protocol.
% go through software network layers and 
% to efficiently bypass these software layers
% We argue that these network layers are \emph{not} inherent to SMR 
% because consensus protocols can already tolerate various faults (\eg, crash in 
% the OS layer). 



% Stat machine replication definition. Focus on networking part, 
% message passing, 
% traditional TCP/IP network. Attractive for general applications, especially 
% online services that put more and more data in memory for speed. Two challenge: 
% slow coordination; no systematic mechanism to practically enforce same 
% execution states among replicas. \xxx, efficient, application agonistic SMR 
% system. \xxx addresses the first challenge by leveraging RDMA to build an 
% speedy paxos protocol. \xxx leverages this protocol to efficiently detect 
% execution divergence that affect network outputs and perform roll back. 

% Evaluation highlights: (1) 10+ general applications, including key-value 
% stores, SQL servers, security servers, ldap servers, and multimedia servers, 
% efficient, little overhead. (2) recover from divergence caused by concurrency 
% bugs, while redis's replication system failed to detect the divergence. (3) 
% with X lines of modifications, faster than redis's own replication system by XX 
% times. 
