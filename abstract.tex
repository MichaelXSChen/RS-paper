
% S1: SMR, reliable, good for online services, machines, network. 
% S2: but going through network round trips for requests hurt latency of 
% services.
% S3: Traditional network layers or OS are not inherent to consensus protocols 
% because these protocols have already considered exceptional cases such as 
% packet losses and kernel failures.
% S4: We can leverage RDMA (one side read/write).
% S5: \xxx, a speed input coordination protocol that levearages fastest RDMA 
% operations.
% \xxx addresses an important challenge that practically enforce execution 
% states on systems nondeterminism such as thread nondeternimism and systems 
% resources contentions.

State machine replication (SMR) runs replicas of the same program and 
enforces same program inputs among replicas with a distributed consensus 
protocol (\eg, \paxos) even facing replica failures. This strong 
fault-tolerance is particularly attractive for building a general SMR system 
for server programs. Unfortunately, to agree on inputs, traditional consensus 
protocols go through software network layers and incur prohibitively latency 
for servers. We present \xxx, a simple, deployable SMR system for general 
server programs by leveraging Remote Direct Memory Access (RDMA) to efficiently 
bypass these software layers. \xxx intercepts inputs in a server's socket API 
and runs a fast consensus protocol that exploits the most efficient RDMA 
one-sided write operations to agree on these inputs. \xxx addresses a pervasive 
challenge, avoiding a server' execution state divergence in active 
replicas, by presenting a fast, application-agnostic output checking mechanism 
on top of this consensus protocol.
% 
% We argue that these network layers are \emph{not} inherent to SMR 
% because consensus protocols can already tolerate various faults (\eg, crash in 
% the OS layer). 

Evaluation on \nprog popular, diverse servers (\eg, \memcached, \mysql, 
and \clamav) shows that \xxx is: (1) general, it ran these servers without any 
modification; (2) fast, it incured merely \overhead mean overhead 
on these programs on popular benchmarks, and it ran 6X$\sim$8X faster than 
three replication systems; and (3) robust, it detected and recovered execution 
divergence caused by a software bug in \redis, while \redis's own replication 
service missed the bug.

% Stat machine replication definition. Focus on networking part, 
% message passing, 
% traditional TCP/IP network. Attractive for general applications, especially 
% online services that put more and more data in memory for speed. Two challenge: 
% slow coordination; no systematic mechanism to practically enforce same 
% execution states among replicas. \xxx, efficient, application agonistic SMR 
% system. \xxx addresses the first challenge by leveraging RDMA to build an 
% speedy paxos protocol. \xxx leverages this protocol to efficiently detect 
% execution divergence that affect network outputs and perform roll back. 

% Evaluation highlights: (1) 10+ general applications, including key-value 
% stores, SQL servers, security servers, ldap servers, and multimedia servers, 
% efficient, little overhead. (2) recover from divergence caused by concurrency 
% bugs, while redis's replication system failed to detect the divergence. (3) 
% with X lines of modifications, faster than redis's own replication system by XX 
% times. 
