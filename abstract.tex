
% S1: SMR, reliable, good for online services, machines, network. 
% S2: but going through network round trips for requests hurt latency of 
% services.
% S3: Traditional network layers or OS are not inherent to consensus protocols 
% because these protocols have already considered exceptional cases such as 
% packet losses and kernel failures.
% S4: We can leverage RDMA (one side read/write).
% S5: \xxx, a speed input coordination protocol that levearages fastest RDMA 
% operations.
% \xxx addresses an important challenge that practically enforce execution 
% states on systems nondeterminism such as thread nondeternimism and systems 
% resources contentions.

State machine replication (SMR) runs replicas of the same program and 
uses a distributed consensus protocol (\eg, \paxos) to enforce same program 
inputs among replicas, tolerating various faults. Recent SMR systems have shown 
to greatly improved the availability of server programs. Unfortunately, 
consensus latency is often too high to make SMR widely adopted. This paper 
presents \xxx, a fast SMR system for general server programs by leveraging 
Remote Direct Memory Access (RDMA). \xxx intercepts a server program's socket 
calls and runs a new RDMA-accelerated \paxos protocol that needs only two 
fastest RDMA write operations per input. This protocol addresses a \paxos 
scalability challenge by tightly integrating RDMA features within the 
fault-tolerant nature of \paxos, making replicas reach consensus rapidly in 
parallel. On top of this protocol, \xxx presents a fast output checking protocol 
to improve assurance on whether replicas run in sync.

% \xxx addresses 
% a pervasive 
% challenge, avoiding a server' execution state divergence in active replicas, by 
% presenting a fast, application-agnostic output checking mechanism on top of our 
% consensus protocol.
% go through software network layers and 
% to efficiently bypass these software layers
% We argue that these network layers are \emph{not} inherent to SMR 
% because consensus protocols can already tolerate various faults (\eg, crash in 
% the OS layer). 

Evaluation on \nprog widely used, diverse server programs (\eg, \memcached and 
\mysql) shows that \xxx is: (1) general, it ran these servers 
without modifications except one program; (2) fast, it incurred merely a 
\latencyoverhead mean overhead in response time and \tputoverhead in 
throughput, and its consensus latency was \fasterthanzookeeper faster than 
a prior SMR system built on \zookeeper; (3) scalable, it achieved similar 
consensus latency from three to seven replicas; and (4) robust, its output 
protocol efficiently detected and recovered replicas from divergence.

% Stat machine replication definition. Focus on networking part, 
% message passing, 
% traditional TCP/IP network. Attractive for general applications, especially 
% online services that put more and more data in memory for speed. Two challenge: 
% slow coordination; no systematic mechanism to practically enforce same 
% execution states among replicas. \xxx, efficient, application agonistic SMR 
% system. \xxx addresses the first challenge by leveraging RDMA to build an 
% speedy paxos protocol. \xxx leverages this protocol to efficiently detect 
% execution divergence that affect network outputs and perform roll back. 

% Evaluation highlights: (1) 10+ general applications, including key-value 
% stores, SQL servers, security servers, ldap servers, and multimedia servers, 
% efficient, little overhead. (2) recover from divergence caused by concurrency 
% bugs, while redis's replication system failed to detect the divergence. (3) 
% with X lines of modifications, faster than redis's own replication system by XX 
% times. 
