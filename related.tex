\section{Related Work} \label{sec:related}


\para{TOCTOU attacks.} TOCTOU (Time-Of-Check-to-Time-Of-Use) 
attacks~\cite{bishop:tocttou,toctou:fast08, toctou:fast05,toctou:usec03} target 
mainly the file system interface, and leverage the non-atomic nature on 
time-of-check (\eg, \texttt{access()}) to time-of-use (\eg, \texttt{open()}) of
a file to gain illegal file access.

A prior concurrency attack study~\cite{con-tr} points out that concurrency 
attacks are much broader and more difficult to track than TOCTOU attacks for 
two main reasons. First, TOCTOU mainly causes illegal file access, while 
concurrency attacks can cause a much broader range of security vulnerabilities, 
ranging from gaining root privileges~\cite{uselib-bug-12791}, injecting 
malicious code~\cite{freebsd-exploit-2009-3527}, to corrupting critical 
memory~\cite{apache-bug-25520}. Second, concurrency attacks are mainly
triggered by concurrent, miscellaneous memory access, thus concurrency attacks 
are much more difficult to track than the file access in TOCTOU attacks.

Our study in this paper differs from prior study~\cite{con-tr} via 
constructing actual inputs and scripts to trigger \nreproduced concurrency 
attacks, and via deeply looking into their input patterns, source-level bug 
patterns, and (in)effectiveness of existing race detection tools. This paper 
also presents \xxx, the first analysis framework to detect concurrency attacks.

\para{Security defense techniques.} Defense techniques for single-threaded
programs have been well studied, including
metadata tracking~\cite{taintdroid:osdi10,
lift:micro06,myers:information,valgrind:pldi}, anomaly
detection~\cite{taskrecycling:ppopp90,schonberg:pldi89}, address space
randomization~\cite{ aslr-fail:ccs04}, and static
analysis~\cite{seth:pldi,engler:vmcai,wagner:syscall:ids,coverity:cacm,
klee:osdi08}.

However, with the presence of multithreading, most existing sequential defense 
tools can be largely weakened or even completely bypassed~\cite{con-tr}. For 
instance, concurrency bugs in global memory may corrupt metadata tags in 
metadata tracking techniques. Anomaly detection lacks a concurrency model to 
reason about concurrency bugs and attacks.
% Static analysis has shown to find 
% numerous bugs in real-world programs, however, when facing a program's huge 
% amount of functions, program paths, and possible thread interleavings, static 
% analysis usually reports too many false positives and bury the real ones 
% (\S\ref{sec:study}).

% \para{Program slicing.}  Program slicing~\cite{Tip:slicing} is a
% general technique to prune irrelevant statements from a program or trace.
% Recently, systems researchers have leveraged or invented slicing
% techniques to block malicious input~\cite{castro:bouncer}, synthesize
% executions for better error diagnosis~\cite{esd:eurosys10}, infer source
% code paths from log messages for postmortem analysis~\cite{sherlog:asplos10}, 
% and identify critical inter-thread reads that may lead to concurrency 
% errors~\cite{conseq:asplos11}.

\para{Concurrency reliability tools.} There are various prior works on 
concurrency
error 
detection~\cite{yu:racetrack:sosp,savage:eraser,racerx:sosp03,lu:muvi:sosp,
avio:asplos06,conmem:asplos10,conseq:asplos11},
diagnosis~\cite{racefuzzer:pldi08,
ctrigger:asplos09,atomfuzzer:fse08}, and
correction~\cite{dimmunix:osdi08,gadara:osdi08,wu:loom:osdi10,cfix:osdi12}.
These works mainly focus on the concurrency bugs themselves, and our \xxx
framework focuses on the security consequence of concurrency 
bugs. Therefore, these works are complementary to \xxx.

Conseq~\cite{conseq:asplos11} is a tool that aims at detecting harmful
concurrency bugs by analyzing its failure consequence. Conseq's main
observation is that concurrency bugs and the bugs' failure sites
are usually within short control and data flow propagation distance (
\eg, within the same function). The concurrency attacks we target in \xxx 
usually exploit corrupted memory that resides in program execution for a long 
time. Thus, the propagations from concurrency bugs to attacks are often long 
(\eg, through different functions). Conseq's proactive harmful thread 
interleaving exploration technique would be useful for \xxx to get more 
potentially vulnerable reports that require tricky thread interleavings to 
trigger. 

Unlike \xxx's forward propagation analysis, Conseq starts form a 
failure site and goes backward to detect concurrency bugs. The reason is 
because Conseq observes that concurrency bugs often have short propagation 
distances to their failure sites. Conseq's analysis works well because: (1) 
this distance is short, and (2) Conseq aims to detect concurrency bugs. 
If the distance is long, backward analysis may have low scalability or 
accuracy. \xxx does forward analysis because we aim to detect 
concurrency attacks from the excessive bug reports and we observed that 
concurrency bugs and their attacks often have long distances (\eg, across 
different functions).

\para{Static vulnerability detection tools.} There are already a 
variety of static vulnerability detection 
approaches~\cite{livshits05finding,yamaguchi:sp14,felmetsger:usec10,
flowdroid:pldi14,srivastava:pldi11,zhang:usenix:sec02} because 
static analysis can thoroughly analyze program code to find vulnerabilities. 
These approaches can be classified into two categories based on whether they 
target general or specific programs.

The first category~\cite{livshits05finding,yamaguchi:sp14} targets general 
programs and their approaches have shown to find severe vulnerabilities in large 
code. However, these pure static analyses may not be adequate to cope with 
concurrency attacks. Benjamin et al.~\cite{livshits05finding} leverages pointer 
analysis to detect data flows from unchecked inputs to sensitive sites. 
This approach ignores control flow and thus it is not suitable to track 
concurrency attacks like the \libsafe one in \S\ref{sec:example}. Yamaguchi 
et al.~\cite{yamaguchi:sp14} did not incorporate inter-procedural analysis and thus 
is not suitable to track concurrency attacks either. Moreover, these general 
approaches are not designed to reason about concurrent behaviors 
(\eg,~\cite{yamaguchi:sp14} can not detect data races).

Our \xxx framework belongs to the first category because it targets general 
programs. Unlike the prior approaches in this category, \xxx incorporates 
concurrency bug detectors to reason about concurrent behaviors, and \xxx's 
consequence analyzer integrates critical dynamic information (\ie, call stacks) 
into static analysis to practically enable the data-flow, control-flow, 
and inter-procedural analysis features.

The second category~\cite{felmetsger:usec10,
flowdroid:pldi14,srivastava:pldi11,zhang:usenix:sec02} lets static analysis 
focus on specific behaviors (\eg, APIs) in specific programs to achieve 
scalability and accuracy. These approaches include checking web application 
logic~\cite{felmetsger:usec10}, Android applications~\cite{flowdroid:pldi14}, 
crossing checking security APIs~\cite{srivastava:pldi11}, and verifying the 
Linux Security Module~\cite{zhang:usenix:sec02}. \xxx's analysis is 
complementary to these approaches; \xxx can be further integrated with these 
approaches to track concurrency attacks in specific scenarios.

% Specific modules/apps.
%   Toward Automated ... web applications: specific web applications.
%   Vetting SSL ...: only for SSL APIs.
%   FlowDroid: leverage Android lifecycle to improve precisions.
%   A security Policy Oracle: spefic API cross checking.
%   CQual: LSM only.
% 
% General programs.
%   Modeling and Discovering ... Property Graphs: can not reason about 
% concurrency (races), intra-procedural.
%   finding Security vulnerabilities ... Static Analysis, Lam: no control flow 
% analysis, may miss the libsafe bugs. Do not handle control-flow indirect 
% propagations.

\noindent {\bf Symbolic execution.} Symbolic execution is an advanced program
analysis technique that can systematically explore a program's execution paths
to find bugs. Researchers have built scalable and effective symbolic execution 
systems to detect software 
bugs~\cite{dart:pldi,cute:fse,godefroid:grammar-fuzzing,
godefroid:whitebox-fuzzing,
klee:osdi08,yang:malicious-disk:oakland06,cadar:exe:ccs06,s2e:hotdep09,
taas:socc10, ucklee:usec15}, block malicious 
inputs~\cite{castro:bouncer}, preserve privacy in error 
reports~\cite{castro:bug-report-privacy}, and detect programming rule 
violations~\cite{woodpecker:asplos13}. Specifically, 
UCKLEE~\cite{ucklee:usec15} has shown to effectively detect hundreds of 
security vulnerabilities in widely used programs. Symbolic execution 
is orthogonal to \xxx; it can be leveraged by \xxx to automatically generate 
programs inputs.
